\documentclass[conference]{IEEEtran}
\IEEEoverridecommandlockouts
% The preceding line is only needed to identify funding in the first footnote. If that is unneeded, please comment it out.
\usepackage{cite}
\usepackage{listings}
\usepackage{amsmath,amssymb,amsfonts}
\usepackage{algorithmic}
\usepackage{graphicx}
\usepackage{textcomp}
\usepackage{xcolor}
\def\BibTeX{{\rm B\kern-.05em{\sc i\kern-.025em b}\kern-.08em
    T\kern-.1667em\lower.7ex\hbox{E}\kern-.125emX}}
\begin{document}

\title{CENG466 - Fundamentals of Image Processing Take Home Exam 2 Report \\
}
\author{\IEEEauthorblockN{1\textsuperscript{st} Fatih Develi}
\IEEEauthorblockA{\textit{Computer Engineering} \\
\textit{Middle East Technical University}\\
Ankara, Turkey \\
fatih.develi@ceng.metu.edu.tr}
\and
\IEEEauthorblockN{2\textsuperscript{nd} Berk Arslan}
\IEEEauthorblockA{\textit{Computer Engineering} \\
\textit{Middle East Technical University}\\
Ankara, Turkey \\
arslan.berk@metu.edu.tr}
}

\maketitle

\begin{abstract}
Frequency domain analysis is an important technique used in image processing. Various
useful applications can be done using this technique. Multiresolution analysis, noise
elimination and edge detection are some of those applications using frequency domain
analysis. This document presents the practice of such applications.
\end{abstract}

\begin{IEEEkeywords}
image processing, image transformation, frequency domain analysis, fourier transform, wavelet
transform, noise reduction, edge detection
\end{IEEEkeywords}

\section{Introduction}
This document is the presentation for the THE2 (Take Home Exam 2) for the
course Fundamentals of Image Processing. Various practices of image transforms were used
to achieve the specified goals. Firstly, wavelet transforms of two images are inspected to
find the wrongly assembled transformation parts and then restore them. Next, fourier transform
of images are used to eliminate the noise in them. Finally, fourier transform is used to
detect the edges in the given images.

Some important code snippets that show the procedures which perform the core image processing operations
are included.

\section{Question 1 - Wavelet Transformations}
In this part, two distorted images whose detail parts of wavelet decompositions are mixed up
are given. The decomposition is done in 3 levels.

The wavelet decomposition of an image can be obtained as following.

% Snippet

Below are the 3 level decompositions of the images.

% Figure

% Figure

As seen in FIGURES, some parts of the wavelet decompositions are swapped and those parts
should be restored to correct places.

After assigning the parts to the relevant places, fixed images are obtained by using the
decompositions of each image.


\section{Question 2 - Noise Elimination}
\subsection{B1}


\subsection{B2}

\subsection{B3}

\subsubsection{Conclusion}


\section{Question 3 - Edge Detection}

\subsection{C1}

\subsection{C3}


\subsection{Conclusion}



\end{document}
